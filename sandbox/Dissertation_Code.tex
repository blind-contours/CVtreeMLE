% Options for packages loaded elsewhere
\PassOptionsToPackage{unicode}{hyperref}
\PassOptionsToPackage{hyphens}{url}
%
\documentclass[
]{article}
\usepackage{amsmath,amssymb}
\usepackage{lmodern}
\usepackage{iftex}
\ifPDFTeX
  \usepackage[T1]{fontenc}
  \usepackage[utf8]{inputenc}
  \usepackage{textcomp} % provide euro and other symbols
\else % if luatex or xetex
  \usepackage{unicode-math}
  \defaultfontfeatures{Scale=MatchLowercase}
  \defaultfontfeatures[\rmfamily]{Ligatures=TeX,Scale=1}
\fi
% Use upquote if available, for straight quotes in verbatim environments
\IfFileExists{upquote.sty}{\usepackage{upquote}}{}
\IfFileExists{microtype.sty}{% use microtype if available
  \usepackage[]{microtype}
  \UseMicrotypeSet[protrusion]{basicmath} % disable protrusion for tt fonts
}{}
\makeatletter
\@ifundefined{KOMAClassName}{% if non-KOMA class
  \IfFileExists{parskip.sty}{%
    \usepackage{parskip}
  }{% else
    \setlength{\parindent}{0pt}
    \setlength{\parskip}{6pt plus 2pt minus 1pt}}
}{% if KOMA class
  \KOMAoptions{parskip=half}}
\makeatother
\usepackage{xcolor}
\usepackage[margin=1in]{geometry}
\usepackage{color}
\usepackage{fancyvrb}
\newcommand{\VerbBar}{|}
\newcommand{\VERB}{\Verb[commandchars=\\\{\}]}
\DefineVerbatimEnvironment{Highlighting}{Verbatim}{commandchars=\\\{\}}
% Add ',fontsize=\small' for more characters per line
\usepackage{framed}
\definecolor{shadecolor}{RGB}{248,248,248}
\newenvironment{Shaded}{\begin{snugshade}}{\end{snugshade}}
\newcommand{\AlertTok}[1]{\textcolor[rgb]{0.94,0.16,0.16}{#1}}
\newcommand{\AnnotationTok}[1]{\textcolor[rgb]{0.56,0.35,0.01}{\textbf{\textit{#1}}}}
\newcommand{\AttributeTok}[1]{\textcolor[rgb]{0.77,0.63,0.00}{#1}}
\newcommand{\BaseNTok}[1]{\textcolor[rgb]{0.00,0.00,0.81}{#1}}
\newcommand{\BuiltInTok}[1]{#1}
\newcommand{\CharTok}[1]{\textcolor[rgb]{0.31,0.60,0.02}{#1}}
\newcommand{\CommentTok}[1]{\textcolor[rgb]{0.56,0.35,0.01}{\textit{#1}}}
\newcommand{\CommentVarTok}[1]{\textcolor[rgb]{0.56,0.35,0.01}{\textbf{\textit{#1}}}}
\newcommand{\ConstantTok}[1]{\textcolor[rgb]{0.00,0.00,0.00}{#1}}
\newcommand{\ControlFlowTok}[1]{\textcolor[rgb]{0.13,0.29,0.53}{\textbf{#1}}}
\newcommand{\DataTypeTok}[1]{\textcolor[rgb]{0.13,0.29,0.53}{#1}}
\newcommand{\DecValTok}[1]{\textcolor[rgb]{0.00,0.00,0.81}{#1}}
\newcommand{\DocumentationTok}[1]{\textcolor[rgb]{0.56,0.35,0.01}{\textbf{\textit{#1}}}}
\newcommand{\ErrorTok}[1]{\textcolor[rgb]{0.64,0.00,0.00}{\textbf{#1}}}
\newcommand{\ExtensionTok}[1]{#1}
\newcommand{\FloatTok}[1]{\textcolor[rgb]{0.00,0.00,0.81}{#1}}
\newcommand{\FunctionTok}[1]{\textcolor[rgb]{0.00,0.00,0.00}{#1}}
\newcommand{\ImportTok}[1]{#1}
\newcommand{\InformationTok}[1]{\textcolor[rgb]{0.56,0.35,0.01}{\textbf{\textit{#1}}}}
\newcommand{\KeywordTok}[1]{\textcolor[rgb]{0.13,0.29,0.53}{\textbf{#1}}}
\newcommand{\NormalTok}[1]{#1}
\newcommand{\OperatorTok}[1]{\textcolor[rgb]{0.81,0.36,0.00}{\textbf{#1}}}
\newcommand{\OtherTok}[1]{\textcolor[rgb]{0.56,0.35,0.01}{#1}}
\newcommand{\PreprocessorTok}[1]{\textcolor[rgb]{0.56,0.35,0.01}{\textit{#1}}}
\newcommand{\RegionMarkerTok}[1]{#1}
\newcommand{\SpecialCharTok}[1]{\textcolor[rgb]{0.00,0.00,0.00}{#1}}
\newcommand{\SpecialStringTok}[1]{\textcolor[rgb]{0.31,0.60,0.02}{#1}}
\newcommand{\StringTok}[1]{\textcolor[rgb]{0.31,0.60,0.02}{#1}}
\newcommand{\VariableTok}[1]{\textcolor[rgb]{0.00,0.00,0.00}{#1}}
\newcommand{\VerbatimStringTok}[1]{\textcolor[rgb]{0.31,0.60,0.02}{#1}}
\newcommand{\WarningTok}[1]{\textcolor[rgb]{0.56,0.35,0.01}{\textbf{\textit{#1}}}}
\usepackage{graphicx}
\makeatletter
\def\maxwidth{\ifdim\Gin@nat@width>\linewidth\linewidth\else\Gin@nat@width\fi}
\def\maxheight{\ifdim\Gin@nat@height>\textheight\textheight\else\Gin@nat@height\fi}
\makeatother
% Scale images if necessary, so that they will not overflow the page
% margins by default, and it is still possible to overwrite the defaults
% using explicit options in \includegraphics[width, height, ...]{}
\setkeys{Gin}{width=\maxwidth,height=\maxheight,keepaspectratio}
% Set default figure placement to htbp
\makeatletter
\def\fps@figure{htbp}
\makeatother
\setlength{\emergencystretch}{3em} % prevent overfull lines
\providecommand{\tightlist}{%
  \setlength{\itemsep}{0pt}\setlength{\parskip}{0pt}}
\setcounter{secnumdepth}{-\maxdimen} % remove section numbering
\usepackage{booktabs}
\usepackage{longtable}
\usepackage{array}
\usepackage{multirow}
\usepackage{wrapfig}
\usepackage{float}
\usepackage{colortbl}
\usepackage{pdflscape}
\usepackage{tabu}
\usepackage{threeparttable}
\usepackage{threeparttablex}
\usepackage[normalem]{ulem}
\usepackage{makecell}
\usepackage{xcolor}
\ifLuaTeX
  \usepackage{selnolig}  % disable illegal ligatures
\fi
\IfFileExists{bookmark.sty}{\usepackage{bookmark}}{\usepackage{hyperref}}
\IfFileExists{xurl.sty}{\usepackage{xurl}}{} % add URL line breaks if available
\urlstyle{same} % disable monospaced font for URLs
\hypersetup{
  pdftitle={Dissertation\_Code},
  pdfauthor={David McCoy},
  hidelinks,
  pdfcreator={LaTeX via pandoc}}

\title{Dissertation\_Code}
\author{David McCoy}
\date{2023-06-22}

\begin{document}
\maketitle

\hypertarget{example}{%
\subsection{Example}\label{example}}

First load the package and other packages needed

\begin{Shaded}
\begin{Highlighting}[]
\FunctionTok{library}\NormalTok{(CVtreeMLE)}
\FunctionTok{library}\NormalTok{(sl3)}
\FunctionTok{library}\NormalTok{(pre)}
\FunctionTok{library}\NormalTok{(partykit)}
\FunctionTok{library}\NormalTok{(kableExtra)}
\FunctionTok{library}\NormalTok{(ggplot2)}

\FunctionTok{set.seed}\NormalTok{(}\DecValTok{429153}\NormalTok{)}
\end{Highlighting}
\end{Shaded}

\hypertarget{simulate-data}{%
\subsection{Simulate Data}\label{simulate-data}}

\begin{Shaded}
\begin{Highlighting}[]
\NormalTok{sim\_data }\OtherTok{\textless{}{-}} \FunctionTok{simulate\_mixture\_cube}\NormalTok{(}
  \AttributeTok{n\_obs =} \DecValTok{800}\NormalTok{,}
  \AttributeTok{splits =} \FunctionTok{c}\NormalTok{(}\FloatTok{0.99}\NormalTok{, }\FloatTok{2.0}\NormalTok{, }\FloatTok{2.5}\NormalTok{),}
  \AttributeTok{mins =} \FunctionTok{c}\NormalTok{(}\DecValTok{0}\NormalTok{, }\DecValTok{0}\NormalTok{, }\DecValTok{0}\NormalTok{),}
  \AttributeTok{maxs =} \FunctionTok{c}\NormalTok{(}\DecValTok{3}\NormalTok{, }\DecValTok{4}\NormalTok{, }\DecValTok{5}\NormalTok{),}
  \AttributeTok{subspace\_assoc\_strength\_betas =} \FunctionTok{c}\NormalTok{(}
    \DecValTok{0}\NormalTok{, }\DecValTok{0}\NormalTok{, }\DecValTok{0}\NormalTok{, }\DecValTok{0}\NormalTok{,}
    \DecValTok{0}\NormalTok{, }\DecValTok{0}\NormalTok{, }\DecValTok{6}\NormalTok{, }\DecValTok{0}
\NormalTok{  )}
\NormalTok{)}
\end{Highlighting}
\end{Shaded}

Let's take a look at the data:

\begin{Shaded}
\begin{Highlighting}[]
\FunctionTok{head}\NormalTok{(sim\_data)}
\end{Highlighting}
\end{Shaded}

\begin{verbatim}
##           age        bmi        sex        M1         M2        M3          y
## 1  0.01651435 -0.4227082 -1.0221195 1.7594922 0.03442708 2.7936966 -0.9910446
## 2  0.19072911  0.4842019  0.9771362 0.1961772 2.34932053 1.3962661  1.1728384
## 3 -0.18790449  0.4828171 -1.0221195 0.4488381 0.04331044 2.6834768 -1.2116326
## 4 -0.19596384 -1.1133632 -1.0221195 0.1387679 2.78777587 0.6990761 -1.2167154
## 5  0.26243848  0.6081797  0.9771362 1.6475103 1.33051234 1.6460804  1.2452379
## 6 -1.32782405 -1.0698419 -1.0221195 1.4097762 0.00406810 2.8495084 -2.3520191
\end{verbatim}

Using the \texttt{simulate\_mixture\_cube} we generate 800 observations
that are exposed to three variables with min values being 0 for all and
max values being 3,4, and 5. In each variable we define split points
0.99, 2.0, and 2.5. Given the eight regions in the cube, the
\texttt{subspace\_assoc\_strength\_betas} parameter is where we put the
outcome in a specific region.

The indices correspond to an area in the cube:

\begin{enumerate}
\def\labelenumi{\arabic{enumi}.}
\tightlist
\item
  All mixtures lower than specified thresholds
\item
  M1 is higher but M2 and M3 are lower
\item
  M2 is higher but M1 and M3 are lower
\item
  M1 and M2 are higher and M3 is lower
\item
  M3 is higher and M1 and M2 are lower
\item
  M1 and M3 are higher and M2 is lower
\item
  M2 and M3 are higher and M1 is lower
\item
  All mixtures are higher than thresholds
\end{enumerate}

So here - we put 6 at index 7 which means the outcome is 6 when M2 and
M3 are higher and M1 is lower than their respective split points. The
outcome is 0 in all other regions.

\hypertarget{run-cvtreemle}{%
\subsection{\texorpdfstring{Run
\texttt{CVtreeMLE}}{Run CVtreeMLE}}\label{run-cvtreemle}}

We will now pass the simulated data and variable names for each node in
O = W,A,Y to the \texttt{CVtreeMLE} function.

\begin{Shaded}
\begin{Highlighting}[]
\NormalTok{ptm }\OtherTok{\textless{}{-}} \FunctionTok{proc.time}\NormalTok{()}

\NormalTok{sim\_results }\OtherTok{\textless{}{-}} \FunctionTok{CVtreeMLE}\NormalTok{(}
  \AttributeTok{data =}\NormalTok{ sim\_data,}
  \AttributeTok{w =} \FunctionTok{c}\NormalTok{(}\StringTok{"age"}\NormalTok{, }\StringTok{"sex"}\NormalTok{, }\StringTok{"bmi"}\NormalTok{),}
  \AttributeTok{a =} \FunctionTok{c}\NormalTok{(}\FunctionTok{paste}\NormalTok{(}\StringTok{"M"}\NormalTok{, }\FunctionTok{seq}\NormalTok{(}\DecValTok{3}\NormalTok{), }\AttributeTok{sep =} \StringTok{""}\NormalTok{)),}
  \AttributeTok{y =} \StringTok{"y"}\NormalTok{,}
  \AttributeTok{n\_folds =} \DecValTok{5}\NormalTok{,}
  \AttributeTok{parallel\_cv =} \ConstantTok{TRUE}\NormalTok{,}
  \AttributeTok{seed =} \DecValTok{2333}\NormalTok{,}
  \AttributeTok{parallel\_type =} \StringTok{"multi\_session"}\NormalTok{,}
  \AttributeTok{family =} \StringTok{"continuous"}\NormalTok{,}
  \AttributeTok{num\_cores =} \DecValTok{6}
\NormalTok{)}

\FunctionTok{proc.time}\NormalTok{() }\SpecialCharTok{{-}}\NormalTok{ ptm}
\end{Highlighting}
\end{Shaded}

\begin{verbatim}
##    user  system elapsed 
##  95.819  24.804 447.939
\end{verbatim}

Note that above, there are default estimators for all parameters if they
are not passed to the function. Here we just use the out of the box
estimators that are defined in \texttt{utils\_create\_sls.R}. These
estimators are chosen to be both non-parametric but also not too
computationally demanding. Examples of estimators used by default are
random forest, xgboost, elastic net, and glms. Users can also pass in
their own custom stacks of learners. We also see here that, using 6
cores with these learners on our simulated data with 800 observations
and 6 variables, our run time is 6 minutes. This can be greatly improved
by increasing the num\_cores parameter.

\hypertarget{results}{%
\subsection{Results}\label{results}}

We can look at the pooled TMLE results for this model. Let's see if
\texttt{CVtreeMLE} identified the current rule in all our folds:

\begin{Shaded}
\begin{Highlighting}[]
\NormalTok{mixture\_results }\OtherTok{\textless{}{-}}\NormalTok{ sim\_results}\SpecialCharTok{$}\StringTok{\textasciigrave{}}\AttributeTok{Pooled TMLE Mixture Results}\StringTok{\textasciigrave{}}
\NormalTok{mixture\_results }\SpecialCharTok{\%\textgreater{}\%}
\NormalTok{  dplyr}\SpecialCharTok{::}\FunctionTok{filter}\NormalTok{(Proportion\_Folds }\SpecialCharTok{==} \FloatTok{1.0}\NormalTok{)}
\end{Highlighting}
\end{Shaded}

\begin{verbatim}
##   Mixture ATE Standard Error Lower CI Upper CI P-value P-value Adj     Vars
## 1       3.259          0.158    2.949    3.568       0           0    M1-M2
## 2       5.935          0.037    5.862    6.007       0           0 M1-M2-M3
##    RMSE
## 1 2.128
## 2 1.069
##                                                                         Union_Rule
## 1                            M1 >= 0.002 & M1 <= 0.966 & M2 >= 1.336 & M2 <= 3.968
## 2 M1 >= 0.002 & M1 <= 0.989 & M2 >= 1.966 & M2 <= 3.968 & M3 >= 2.436 & M3 <= 4.99
##   Proportion_Folds
## 1                1
## 2                1
\end{verbatim}

Above, the estimated mixture ATE for this rule is 5.94 (5.84 - 6.03),
which covers our true mixture ATE used to generate the data which was 6.
The estimated mixture ATE is interpreted as: the average counterfactual
mean outcome if all individuals were exposed to the rule shown in
\texttt{Union\ Rule} compared to if all individuals were unexposed is
5.94. That is, those individuals who are exposed to this rule have an
outcome that is 5.94 higher compared to those that are not exposed to
this rule. The standard error, confidence intervals and p-values are
derived from the influence curve of this estimator.

We can also look at the v-fold specific results. This gives the analyst
the ability to investigate how stable the estimates and rules are. These
results are the same as standard sample splitting techniques and
therefore have proper variance estimates and p-values. Below we show the
v-fold specific interactions found with fold specific estimates of our
ATE target parameter and variance estimates from the fold specific IC.

\begin{Shaded}
\begin{Highlighting}[]
\NormalTok{mixture\_v\_results }\OtherTok{\textless{}{-}}\NormalTok{ sim\_results}\SpecialCharTok{$}\StringTok{\textasciigrave{}}\AttributeTok{V{-}Specific Mix Results}\StringTok{\textasciigrave{}}
\NormalTok{mixture\_v\_results}\SpecialCharTok{$}\StringTok{\textasciigrave{}}\AttributeTok{M1{-}M2{-}M3}\StringTok{\textasciigrave{}}
\end{Highlighting}
\end{Shaded}

\begin{verbatim}
##     ate    se lower_ci upper_ci p_val p_val_adj  rmse
## 1 5.893 0.066   5.7630   6.0230     0         0 1.184
## 2 5.946 0.043   5.8610   6.0300     0         0 0.997
## 3 5.946 0.109   5.7320   6.1600     0         0 1.178
## 4 5.940 0.114   5.7160   6.1630     0         0 1.300
## 5 5.948 0.071   5.8090   6.0880     0         0 1.113
## 6 5.935 0.190   5.5627   6.3077     0         0 1.089
##                                                                           mix_rule
## 1                                            M3 > 2.468 & M2 > 1.975 & M1 <= 0.986
## 2                                            M3 > 2.481 & M1 <= 0.995 & M2 > 1.975
## 3                                            M2 > 2.006 & M3 > 2.408 & M1 <= 0.985
## 4                                            M3 > 2.481 & M2 > 1.966 & M1 <= 0.986
## 5                                            M3 > 2.481 & M2 > 1.975 & M1 <= 0.989
## 6 M1 >= 0.002 & M1 <= 0.989 & M2 >= 1.966 & M2 <= 3.968 & M3 >= 2.436 & M3 <= 4.99
##     fold variables
## 1      1  M1-M2-M3
## 2      2  M1-M2-M3
## 3      3  M1-M2-M3
## 4      4  M1-M2-M3
## 5      5  M1-M2-M3
## 6 Pooled  M1-M2-M3
\end{verbatim}

In v-fold specific results we also give a pooled estimate. This is
different than the pooled TMLE estimate. Here we simply take the
weighted average of the fold specific ATEs and the harmonic mean of the
variances. This is similar to meta-analysis approaches.

We can plot our v-fold mixture results findings using the
\texttt{plot\_mixture\_results} function. This will return a list of
plots with names corresponding to the interactions found.

This plot shows the ATE specific for each fold and for the weighted-mean
results over the fold with corresponding pooled variance. The rule is
the union rule which includes all observations that were indicated by
the fold specific rules.

\texttt{CVtreeMLE} also data-adaptively identifies thresholds in the
marginal space. This feature is described in the vignette. In the
marginal setting, partitions are found for each mixture variable
individually and the ATE is in reference to the baseline (lowest leaf)
value.

Additional details for this and other features are given in the
vignette.

\end{document}
